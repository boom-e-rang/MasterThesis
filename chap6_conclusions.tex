
A variety of shared control paradigms have been implemented to provide feedback and assistance to users in settings where the task is known \textit{a priori}. Although users might prefer to maintain control and user engagement is necessary for learning, many applications require a certain level of control authority to be allocated to the machine in order to guarantee safety, improve successful task completion, and/or accelerate learning. As such, most interfaces employ support strategies that in various ways restrict or adjust users' actions in order to enable the subject and the device to move in a safe and constructive manner. 

In this work, we present and evaluate an assessment criterion for user input that can be used in these shared control paradigms. We carry out experiments by using the MIG as an evaluation criterion in a filter-based shared control scheme, similar to \cite{hapticsharedcontrol_review, MDA_Katie}, where user actions deemed by the filter as incorrect are either blocked or hindered by the hardware interface. With only current state information, our proposed filter can both reject unhelpful inputs and remain transparent to operators with significant skill avoiding unnecessary interference improving safety or accelerating training. 

An important feature of the presented interface is its adaptability in real-time. It requires no predefined trajectory, runs on an indefinite time horizon, and automatically adapts to operator skill. It can, like existing adaptive methods, enhance performance while avoiding some of the common long-term pitfalls of ``static" automation such as over-reliance, skill degradation, and reduced situation awareness \cite{adaptive_review}. For complex tasks that require human operators, such as operating a crane or flying an aircraft, these features can alleviate or minimize the need for training with virtual simulators by ensuring safety of the physical system in action. For therapeutic applications, a MIG MDA interface may accelerate recovery or provide assistance by preventing patient slacking, relieving frustration, and utilizing intentioned but noisy signals (e.g., tremor and spasticity) of patients with sensorimotor disorders. Particularly during support in dynamic tasks, such as walking with an exoskeleton, the algorithm can help provide meaningful assistance and ensure safety of the operator-machine system without limiting the user's freedom to maneouver while coupled with the device. 

In future work, the proposed evaluation criterion and filter-based shared control scheme should be explored further. Experimental and theoretical work with MDA in assistance mode should be conducted to establish whether guarantees can be made about the safety of the human-machine system. Further experiments should be carried out to make direct comparisons of MIG MDA to other assist-as-needed controllers. Additionally, we would like to explore other objective functions, such as ergodic objectives or barrier functions, instead of currently used quadratic cost on state to test whether we can make the interface even less restrictive to the users' chosen strategies. At the same time, subjects should be tested in higher dimensional tasks to see if all features of the paradigm hold true even when task complexity increases. And lastly, studies of impaired subjects training in clinically relevant tasks should be performed. 

